\documentclass[a4paper,12pt,twoside]{article}
\usepackage[T1,T2A]{fontenc}
\usepackage[utf8]{inputenc}


\usepackage{graphicx}
\usepackage{blindtext}
\usepackage{mathrsfs}
\usepackage[russian,english]{babel}
\usepackage{enumitem}
\usepackage[english,russian]{babel}
\usepackage{amsmath}
\usepackage[section]{placeins}
\usepackage{amssymb}
\usepackage[top=20mm, bottom=20mm, left=20mm, right=20mm]{geometry}

\begin{document}
	\title{Модель $X \boxplus K$}
	\maketitle
	
	\section{ Математическое описание моделей}
	
	Определим математичские модели "$X \boxplus K$":
	
	\begin{enumerate}
	\item 
	$Y_{g_4} = g_4(x) \equiv x \boxplus K,$ где
	
	$x \in V_{4}$ - вектор входных данных,
	
	$k \in V_{4}$ - некоторый неизвестный постоянный в эксперименте ключ,
	
	$Y_{g_4} \in V_{4}$ - выходные данные модели $g_4$;
	\bigskip
		
	\item 
	$Y_{g_8} = g_8(x) \equiv x \boxplus K,$ где
	
	$x \in V_{8}$ - вектор входных данных,
	
	$k \in V_{8}$ - некоторый неизвестный постоянный в эксперименте ключ,
	
	$Y_{g_8} \in V_{8}$ - выходные данные модели $g_8$;
	\bigskip
	
		\item 
	$Y_{g_{16}} = g_{16}(x) \equiv x \boxplus K,$ где
	
	$x \in V_{16}$ - вектор входных данных,
	
	$k \in V_{16}$ - некоторый неизвестный постоянный в эксперименте ключ,
	
	$Y_{g_{16}} \in V_{16}$ - выходные данные модели $g_{16}$;
	\bigskip
	
	\item 
	$Y_{g_{32}} = g_{32}(x) \equiv x \boxplus K,$ где
	
	$x \in V_{32}$ - вектор входных данных,
	
	$k \in V_{32}$ - некоторый неизвестный постоянный в эксперименте ключ,
	
	$Y_{g_{32}} \in V_{g_{32}}$ - выходные данные модели $g_{32}$.
	\bigskip

	\end{enumerate}

	
	\newpage
	\section{Описание используемых нейронных сетей}	
	\bigskip
	\noindent Для решения поставленных задач использовались следующие нейронные сети (с минимальным количеством параметров):
	
	\begin{enumerate}
		\item Модель $g_4$: нейронная сеть с одним скрытым слоем, с 4 нейронами на скрытом слое (HNN-4).
		\item Модель $g_8$: нерйонная сеть с одним скрытым слоем, с 8 нейронами на скрытом слое (HNN-8).
		\item Модель $g_{16}$: нерйонная сеть с одним скрытым слоем, с 16 нейронами на скрытом слое (HNN-16).
		\item Модель $g_{32}$: нерйонная сеть с одним скрытым слоем, с 64 нейронами на скрытом слое (HNN-64).
		(Примечание: для модлеи $g_{32}$ также была построенна НС с 32 нейронами на скрытом слое. Однако масимально достигнутая точность была 87.5\%. Поэтому от нее я отказался и использовал модель HNN-64.)
	\end{enumerate}
	

	\noindent Точность построенной модели к реальной оценивалась использовалось расстояние Хэмминга:
	$w(y, \hat{y}) = \sum_{i=1}^{j} y_i \oplus \hat{y}_i$, где $y_i \in V_j$.
	\bigskip
	
	\noindent Для оценки точности проведенного эксперимента использовалось следующая функция:
	$\hat{f} =L - \dfrac{1}{T_e}\sum_{j=1}^{T_e}w(y^{(j)}, \hat{y^{(j)}})$,
	где $L$ - количество бит в выходных данных оцениваемой модели. 

	\bigskip
	\noindent Компьютерные эксперименты проводились на следующих данных:
	\begin{enumerate}
		\item Модель $G_4$ :
		\begin{itemize}
		\item количество параметров для HNN-4: 32; 
		\item обучающая выборка $T_o=10$ пар $(x, y)$;
		\item экзаменационная выборка $T_e=6$ пар $(x, y)$.
		\end{itemize}
	
		\item Модель $G_8$:
		\begin{itemize}
			\item количество параметров для HNN-8: 128; 
			\item обучающая выборка $T_o=128$ пар $(x, y)$;
			\item экзаменационная выборка $T_e=24$ пар $(x, y)$.
		\end{itemize}
	
		\item Модель $G_{16}$:
		\begin{itemize}
			\item количество параметров для HNN-16: 512; 
			\item обучающая выборка $T_o=512$ пар $(x, y)$;
			\item экзаменационная выборка $T_e=102$ пар $(x, y)$.
		\end{itemize}
		
		\item Модель $G_{32}$:
		\begin{itemize}
			\item количество параметров для HNN-64: 4096; 
			\item обучающая выборка $T_o=2048$ пар $(x, y)$;
			\item экзаменационная выборка $T_e=409$ пар $(x, y)$.
		\end{itemize}
	\end{enumerate}
	
	
	\newpage
	\section{Результаты компьютерных экспериментов}
	
	\begin{figure}[htb!]
	\includegraphics[width=0.8\linewidth]{report_g4/g4.png}
	
	График точности построенной неройнной сети HNN-4 модели $g_4$ от количества итераций обучения.
		
	\end{figure}
	\begin{figure}[htb!]
	\includegraphics[width=0.8\linewidth]{report_g4/g8.png}
	
	График точности построенной неройнной сети HNN-8 модели $g_8$ от количества итераций обучения.
	
	\end{figure}

		\begin{figure}[htb!]
		\includegraphics[width=0.8\linewidth]{report_g4/g16.png}
		
		График точности построенной неройнной сети HNN-16 модели $g_{16}$ от количества итераций обучения.
		
	\end{figure}

	\begin{figure}[htb!]
	\includegraphics[width=0.8\linewidth]{report_g4/g32.png}
	
	График точности построенной неройнной сети HNN-64 модели $g_{32}$ от количества итераций обучения.
	
\end{figure}

		
\end{document}