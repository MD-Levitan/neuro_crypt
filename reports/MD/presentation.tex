\documentclass[notheorems]{beamer}

\usepackage{style/bsupresentation} % Подгружаем тему

%%% Работа с русским языком и шрифтами
\usepackage[english,russian]{babel}   % загружает пакет многоязыковой вёрстки
%%\usepackage{fontspec}      % подготавливает загрузку шрифтов Open Type, True Type и др.
%\defaultfontfeatures{Ligatures={TeX},Renderer=Basic}  % свойства шрифтов по умолчанию
%\setmainfont[Ligatures={TeX,Historic}]{Arial} %TODO: Helvetica Light Normal
%\setsansfont{Arial}  %TODO: Helvetica Light Normal
%\setmonofont{Courier New}

\usepackage{tikz}
\usetikzlibrary{shapes,arrows}

% Оформление теорем, лемм и т.д.
\theoremstyle{plain}
\newtheorem{Theorem}{Теорема}[]
\newtheorem{Lemma}{Лемма}[]
\newtheorem{Proposition}{Предложение}[]
\newtheorem{Corollary}{Следствие}[]
\newtheorem{Statement}{Утверждение}[]

\theoremstyle{definition}
\newtheorem{Definition}{Определение}[]
\newtheorem{Conjecture}{Гипотеза}[]
\newtheorem{Algorithm}{Алгоритм}[]
\newtheorem{Property}{Свойство}[]

\theoremstyle{remark}
\newtheorem{Remark}{Замечание}[]
\newtheorem{Example}{Пример}[]
\newtheorem{Note}{Примечание}[]
\newtheorem{Case}{Случай}[]
\newtheorem{Assumption}{Предположение}[]

\uselanguage{russian}
\languagepath{russian}
\deftranslation[to=russian]{Theorem}{Теорема}
\deftranslation[to=russian]{Lemma}{Лемма}
\deftranslation[to=russian]{Definition}{Определение}
\deftranslation[to=russian]{Definitions}{Определения}
\deftranslation[to=russian]{Corollary}{Следствие}
\deftranslation[to=russian]{Fact}{Факт}
\deftranslation[to=russian]{Example}{Пример}
\deftranslation[to=russian]{Examples}{Примеры}

\usepackage{multicol}       % Несколько колонок
\graphicspath{{graphics/}}    % Папка с картинками

%%% Информация об авторе и выступлении
\title[]{Оценивание надежности криптографических преобразований на основе нейронных сетей}
\author[М.Ю. Деркач]{Максим Юрьевич Деркач\\ \smallskip Научный руководитель: Юрий Семенович Харин}
\institute[ММАД, ФПМИ, БГУ]{Факультет прикладной математики и информатики \\ \smallskip Кафедра математического моделирования и анализа данных}
\date{Минск, \the\year}

\begin{document}

\frame[plain]{\titlepage}

\begin{frame}
  \frametitle{Содержание}
  \tableofcontents
\end{frame}

\section{Введение}
\begin{frame}
\begin{scriptsize}
  \frametitle{\secname}
  \framesubtitle{Обзор литературы}
  \begin{thebibliography}{5}
    \beamertemplatearticlebibitems
    \bibitem{crypto} Харин Ю.С., Берник В.И., Матвеев Г.В., Агиевич С.В. Математические и компьютерные основы криптологии. -- 2003. -- Минск.
    \beamertemplatearticlebibitems
    \bibitem{neuro_des} Mohammed M. Alani.Neuro-Cryptanalysis of DES and Triple-DES. -- 2012. 
    \beamertemplatearticlebibitems
    \bibitem{kinzel} Kinzel F., Kanter I. Neural Cryptography. -- 2002.
    \beamertemplatearticlebibitems
    \bibitem{encr_nn} Pattanayak S., Ludwig S.A. Encryption based on Neural Cryptography. -- 2017.
    \beamertemplatearticlebibitems
    \bibitem{bitwise_nn}M. Kim, P. Smaragdis, Bitwise Neural Networks. -- 2010.
    
  \end{thebibliography}
\end{scriptsize}
\end{frame}


\begin{frame}
	\frametitle{\secname}
	\framesubtitle{Цель исследования и постановка задач}
	Цель исследования - оценивание надежности криптографических преобразований сети Фейстеля, используя искусственные нейронные сети.
\end{frame}

\begin{frame}
	\frametitle{\secname}
	\framesubtitle{Цель исследования и постановка задач}
Задачи:
\begin{enumerate}
	\item Провести аналитический анализ работ по теме нейронные сети в криптографии.

	\item Основываясь на задачах исследования, определить архитектуру и параметры нейронной сети.
	
	\item Определить математические модели криптографических примитивов преобразования Фейстеля.
	
	\item Определить математические модель криптографического преобразования Фейстеля.
	
	\item Разработать генератор модельных данных.
	
	\item Построить нейронные сети, аппроксимирующие данную математическую модель. Провести компьютерные эксперименты.
	
	\item Оценить результаты полученные в ходе компьютерных экспериментов.
\end{enumerate}
\end{frame}

\section{Искусственные нейронные сети и их применение в криптографии}
\begin{frame}
  \frametitle{\secname}
  \framesubtitle{Модель криптоанализа}
  
  \tikzset{%
  	block/.style    = {draw, thick, rectangle, minimum height = 3em,
  		minimum width = 3em},
  	sum/.style      = {draw, circle, node distance = 2cm}, % Adder
  	input/.style    = {coordinate}, % Input
  	output/.style   = {coordinate} % Output
  }
  
  \begin{figure}[H]
  	\begin{tikzpicture}[auto, thick, node distance=2cm, >=triangle 45, scale=\textwidth/15.2cm]
  	\draw
  	
  	node at (0,0)[right=-3mm]{\Large$\bullet$}
  	node [input, name=plain] {} 
  	node at (0,-3)[right=-3mm]{\Large$\bullet$}
  	node at (0,-3) [input, name=cipher] {} 
  	
  	node at (5,-3)[block] (nn) {$Neural\ Network$}
  	node at (10, 0)[block] (err) {$Error\ Function$}
  	
  	node at (10,-6)[block] (weight) {$Weights\ Correction$};
  	
  	
  	\draw[->](plain) -- node {$Ciphertext$} (err);
  	\draw[->](cipher) -- node {$Plaintext$} (nn);
  	\draw[->](nn) -- node {} (err);
  	
  	\draw[->](err) -- node {} (weight);
  	\draw[->](weight) -- node {} (nn);
  	
  	\end{tikzpicture}
  	\caption{Модель криптоанализа на основе нейронной сети}
  \end{figure}
\end{frame}

\begin{frame}
	\frametitle{\secname}
	\framesubtitle{Оценка архитектуры и параметров нейронных сетей}
	


\end{frame}

\section{Аппроксимация криптографических примитивов преобразования Фейстеля}
\begin{frame}
	\frametitle{\secname}
	\framesubtitle{Математические модели}

	Опишем однотактовое преобразование шифрования ГОСТ 28147-89:
	\bigskip
	
	$Y = g(X, K) = g(X_1 || X_2, K) \equiv (S[X_1 \boxplus K] \ll 11) \oplus X_2 || X_1,$ где
	
	$X \in V_{64}$ - вектор входных данных,
	
	$Y \in V_{64}$ - выходные данные,
	
	$K \in V_{32}$ - ключ,
	
	$S$ - стандартный S-блок из ГОСТ-28147.
		
\end{frame}


\begin{frame}
	\frametitle{\secname}
	\framesubtitle{Математические модели}
	
\begin{enumerate}
	\item
	$Y_{g_0} = g_0(x) = g_0(x_1 || x_2)\equiv x_1 \oplus x_2,$ где
	
	$x \in V_{8}$ - вектор входных данных,
	
	$x_1, x_2 \in V_{4}$ - левая и правая часть входного вектора,
	
	$Y_{g_0} \in V_{4}$ - выходные данные модели $g_0$;
	\bigskip
	
	\item
	$Y_{g_1} = g_1(x) = g_1(x_1 || x_2, k) \equiv S[x_1] \oplus x_2,$ где
	
	$x \in V_{8}$ - вектор входных данных,
	
	$x_1, x_2 \in V{4}$ - левая и правая часть входного вектора,
	
	$Y_{g_1} \in V_{4}$ - выходные данные модели $g_1$,
	
	$S$ - первый узел из стандартного S-блока из ГОСТ-28147 
	
	($ S = \{13, 2, 8, 4, 6, 15, 11, 1, 10, 9, 3, 14, 5, 0, 12, 7\}$);
	\bigskip
	
	
\end{enumerate}
	
\end{frame}


\begin{frame}
	\frametitle{\secname}
	\framesubtitle{Математические модели}
	
	\begin{enumerate}
		\setcounter{enumi}{2}
		\item
		$Y_{g_2} = g_2(x) = g_2(x_1 || x_2, k) \equiv S[x_1 \boxplus k] \oplus x_2,$ где
		
		$x \in V_{8}$ - вектор входных данных,
		
		$x_1, x_2 \in V{4}$ - левая и правая часть входного вектора,
		
		$k \in V_{4}$ - некоторый неизвестный постоянный в эксперименте ключ,
		
		$Y_{g_2} \in V_{4}$ - выходные данные модели $g_2$,
		
		$S$ - первый узел из стандартного S-блока из ГОСТ-28147;
		
		\item 
		$Y_{g_3} = g_3(X) = g_3(x_1 || x_2, k) \equiv (S[x_1 \boxplus k] \ll 11) \oplus x_2,$ где
		
		$x \in V_{64}$ - вектор входных данных,
		
		$k \in V_{32}$ - некоторый неизвестный постоянный в эксперименте ключ,
		
		$Y_{g_3} \in V_{32}$ - выходные данные модели $g_3$,
		
		$S$ - стандартный S-блок из ГОСТ-28147;
		
		
	\end{enumerate}
	
\end{frame}

\begin{frame}
	\frametitle{\secname}
	\framesubtitle{Математические модели}
	
	\begin{enumerate}
		\setcounter{enumi}{4}
		
		\item
		$Y_{g_4} = g_4(x) \equiv x \boxplus K,$ где
		
		$x \in V_{4}$ - вектор входных данных,
		
		$k \in V_{4}$ - некоторый неизвестный постоянный в эксперименте ключ,
		
		$Y_{g_4} \in V_{4}$ - выходные данные модели $g_4$;
		\bigskip
		
		\item
		$Y_{g_8} = g_8(x) \equiv x \boxplus K,$ где
		
		$x \in V_{8}$ - вектор входных данных,
		
		$k \in V_{8}$ - некоторый неизвестный постоянный в эксперименте ключ,
		
		$Y_{g_8} \in V_{8}$ - выходные данные модели $g_8$;
		\bigskip
	
		
	\end{enumerate}
	
\end{frame}

\begin{frame}
	\frametitle{\secname}
	\framesubtitle{Математические модели}
	
	\begin{enumerate}
		\setcounter{enumi}{6}
		
		
		\item
		$Y_{g_{16}} = g_{16}(x) \equiv x \boxplus K,$ где
		
		$x \in V_{16}$ - вектор входных данных,
		
		$k \in V_{16}$ - некоторый неизвестный постоянный в эксперименте ключ,
		
		$Y_{g_{16}} \in V_{16}$ - выходные данные модели $g_{16}$;
		\bigskip
		
		\item
		$Y_{g_{32}} = g_{32}(x) \equiv x \boxplus K,$ где
		
		$x \in V_{32}$ - вектор входных данных,
		
		$k \in V_{32}$ - некоторый неизвестный постоянный в эксперименте ключ,
		
		$Y_{g_{32}} \in V_{g_{32}}$ - выходные данные модели $g_{32}$.
		\bigskip
		
	\end{enumerate}
	
\end{frame}



\begin{frame}
  \frametitle{\secname}
  \framesubtitle{Численные результаты}
  
  \begin{figure}[H]
  	\includegraphics[width=0.7\linewidth]{results/g0_0.png}
  	
  	\caption{График точности построенной однослойной нейронной сети модели $g_0$ от количества итераций обучения}
  	
  \end{figure}

\end{frame}


\begin{frame}
	\frametitle{\secname}
	\framesubtitle{Численные результаты}

	\begin{figure}[H]
		\includegraphics[width=0.8\linewidth]{results/g0_1.png}
		
		\caption{График точности построенной нейронной сети с одним скрытым слоем модели $g_0$ от количества нейронов на скрытом слое}
	\end{figure}

\end{frame}


\begin{frame}
	\frametitle{\secname}
	\framesubtitle{Численные результаты}
	
	
	\begin{figure}[H]
		\includegraphics[width=0.8\linewidth]{results/g0_1_x.png}
		
		\caption{График точности построенной нейронной сети с одним скрытым слоем и 8 нейронами на нем модели $g_0$ от количества нейронов на скрытом слое}
	\end{figure}
	
\end{frame}


\begin{frame}
	\frametitle{\secname}
	\framesubtitle{Численные результаты}
	
\begin{table}[H]
	\scalebox{0.7}{%
	\begin{tabular}{|l|l|l|l|l|l|l|l|l|l|}
		\hline
		Модель  & НС    & $N$         & $NP$              & $T_o$       & $T_e$  & $\hat{f}$ & l   & $\Sigma$ & $\Psi$ \\ \hline
		$g_{0}$ & MNN   &      8      &            96     &         128 &    24  &   4       &  0.1072   & 5000     &   1.3754    \\ \hline
		$g_{0}$ & NN    &      0      &            96     &         128 &    24  &   3.2     &  0.5867   & 5000     &   1.6164    \\ \hline
		$g_{1}$ & MNN   &      8      &            96     &         128 &    24  &   4       &  0.1365   & 10000    &   2.7018    \\ \hline
		$g_{1}$ & NN    &      0      &            96     &         128 &    24  &   2.300   &  0.6505   & 25000    &   6.1263    \\ \hline
		$g_{2}$ & MNN   &      8      &            96     &         128 &    24  &   4       &  0.1585   & 20000    &   5.4420    \\ \hline
		$g_{2}$ & NN    &      0      &            96     &         128 &    24  &   2.6539  &  0.6341   & 20000    &   7.1460 	\\ \hline
		%$g_{3}$ & MNN   &      32       &                   &         2048&   409  &           &     & 10000    &       \\ \hline
		%$g_{3}$ & NN    &             &                   &         2048&   409  &           &     & 10000    &       \\ \hline
		$g_{4}$ & MNN   &      4      &            32     &         10  &     5  &    4      &  0.0720   & 10000    &   2.0639    \\ \hline
		$g_{4}$ & NN    &      0      &            32     &         10  &     5  &    3.500  &  0.5526   & 10000    &   3.1317    \\ \hline
		$g_{8}$ & MNN   &      8      &            128    &         128 &    24  &    8      &  0.0519   & 5000     &   2.4865    \\ \hline
		$g_{8}$ & NN    &      0      &            128    &         128 &    24  &    7.5385 &  0.5039   & 10000    &  3.8590    \\ \hline
		$g_{16}$& MNN   &      16     &            512    &         512 &   129  &   16      &  0.0520   & 15000    &  6.3108     \\ \hline
		$g_{16}$& NN    &      0     &            512    &         512 &   129  &   14.3650 &  0.5386   & 20000    &  9.5851    \\ \hline
		$g_{32}$& MNN   &      32     &            4096   &         2048&   409  &   32      &  0.0120   & 10000    &  18.8175     \\ \hline
		$g_{32}$& NN    &      0     &            4096   &         2048&   409  &   27.2161 &  0.5614   & 15000    &  20.2561     \\ \hline
	\end{tabular}}
	\caption{Сравнение построенных нейронных сетей}
\end{table} 
	
\end{frame}


\section{Оценка надежности криптографического преобразования Фейстеля с помощью его аппроксимации нейронной сетью}
\begin{frame}
	\frametitle{\secname}
	\framesubtitle{Математические модели}
	
	$V = {0, 1};$
	
	$N$- размерность;
	
	$X = (x_i) \in V^{2N}, X = (X_1 \| X_2) \in V^{2N}, X_i \in V^N, i=1,2;$
	
	
	$Y = (y_i) \in V^{2N}, Y = (Y_1 \| Y_2) \in V^{2N}, Y_i \in V^N, i=1,2;$
	
	
	$K=(K1\|...\|K_8)=(k_1,...,k_{8N})$ - ключ тактового преобразования;
	
	
	$<K_i> \in \{0,1,...,2^{N}-1\}$ - числовое представление двоичного вектора $K_i$;
	
	
	$\lll L$ - циклический сдвиг влево на $L$ бит;
	
	
	$f(\cdot):V^N \times V^N \rightarrow V^N$ - функция криптографического преобразования Фейстеля;
	
	
	$g(\cdot):V^N \rightarrow V^N$ - расписание ключей.
	
\end{frame}

\begin{frame}
	\frametitle{\secname}
	\framesubtitle{Математические модели}
	
	  \tikzset{%
		block/.style    = {draw, thick, rectangle, minimum height = 0.5em,
			minimum width = 0.5em},
		sum/.style      = {draw, circle, node distance = 1cm}, % Adder
		input/.style    = {coordinate}, % Input
		output/.style   = {coordinate} % Output
	}
\begin{figure}
	\begin{tikzpicture}[auto, thick, node distance=2cm, >=triangle 45, scale=\textwidth/26.2cm]
	\draw
	% Drawing the blocks of first filter :
	node [block] (x1) {$X_1$}
	node at (10,0)[block] (x2) {$X_2$}
	
	node at (10,-3)[sum] (f) {$f(\cdot)$}
	node at (13,-3)[sum] (k) {$K_i$}
	
	node at (10,-6)[block] (l) {$\lll L$}
	node at (10,-9)[sum] (sum) {$\oplus$}
	
	node at (0,-12) [block] (y1) {$Y_1$}
	node at (10,-12)[block] (y2) {$Y_2$};
	% Joining blocks. 
	
	\draw[->](x1) -- node {} (sum);
	\draw[->](x2) -- node {} (f);
	\draw[->](x2) -- node {} (y1);
	\draw[->](k) -- node {} (f);
	\draw[->](f) -- node {} (l);
	\draw[->](l) -- node {} (sum);
	\draw[->](sum) -- node {} (y2);
	
	
	\end{tikzpicture}
	\caption{Модельное 1-тактовое преобразование с $i$-ым подключом}
\end{figure}
	
\end{frame}


\begin{frame}
	\frametitle{\secname}
	\framesubtitle{Математические модели}

	$f(X_2;K) = S[X_2 \boxplus K]$,
	где $S$ - два первых стандартных S-блока из ГОСТ-28147.
	
	$g(K)=K_1, ..., K_8; K_1,...K_8;...;K_1,...K_8$.
	
	Параметр $N$ постоянный и равен = 8.
	
	Параметры $L$, $I$ - изменяемые параметры в следующих диапозонах:
	\begin{itemize}
		\item $L \in \{0,1,...,7\};$
		
		\item $I \in \{1,...,8\};$
	\end{itemize}
	
\end{frame}


\begin{frame}
	\frametitle{\secname}
	\framesubtitle{Численные результаты}
	
	
\end{frame}

\section{Заключение}
\begin{frame}
  \frametitle{\secname}
  Основные результаты:
  \begin{enumerate}
    \item Для математических моделей криптографических преборазований Фейстеля был разработан генератор и построенны нейронные сети;
    \item Была проведена оценка полученных результатов.
    \item Проведены компьютерные эксперименты, иллюстрирующие теоретические результаты.
  \end{enumerate}
\end{frame}

\begin{frame}[c]
\begin{center}
\frametitle{\LARGE Спасибо за внимание!}

\bigskip

\inserttitle

\bigskip\bigskip
\insertauthor

\bigskip\bigskip\bigskip\bigskip\bigskip\bigskip

\insertdate

\end{center}
\end{frame}

\end{document}