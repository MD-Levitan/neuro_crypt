  \chapter*{Введение}
  \addcontentsline{toc}{chapter}{Введение}
  \label{c:intro}
  
Проблема защиты информации затрагивает практически все сферы деятельности человека. И среди способов защиты информации важнейшим считается криптографический \cite{crypto}.

\bigskip

Нейронная криптография "--- это раздел криптографии, посвященный анализу применения стохастических алгоритмов, особенно алгоритмов искусственных нейронных сетей, для использования в шифровании и криптоанализе \cite{crypto_nn}.

\bigskip

Идея использовать нейронные сети в криптографии нова. Впервые она была озвучена Себастьяном Дорленсом в 1995 году, спустя 30 лет после
определения основ нейронных сетей. В криптоанализе используется способность нейронных сетей исследовать пространство решений. Также
имеется возможность создавать новые типы атак на существующие алгоритмы шифрования, основанные на том, что любая функция может быть
представлена нейронной сетью. Взломав алгоритм, можно найти решение, по крайней мере, теоретически. При этом используются такие свойства
нейронных сетей, как взаимное обучение, самообучение и стохастическое поведение, а также низкая чувствительность к шуму, неточностям (искажения
данных, весовых коэффициентов, ошибки в программе). Также архитектура искусственных нейронных сетей позволяет эффективно проводить работы по распознаванию образов и классификации множества объектов по любым признакам. Кроме того, благодаря хорошо продуманному алгоритму обученные нейронные сети могут достигать чрезвычайно высоких уровней точности. Они позволяют решать проблемы криптографии с открытым ключом, распределения ключей,
хэширования и генерации псевдослучайных чисел.\cite{encr_nn}.

\bigskip

В данной работе будет рассмотрены задачи, связанные с оценкой надежности криптографических преобразований. Одной из этих задач является задача оценки стойкости итерационного преобразования Фейстеля, которая имеет важное теоретическое и практическое значение. Задача находит приложения в области криптографии и защиты данных.