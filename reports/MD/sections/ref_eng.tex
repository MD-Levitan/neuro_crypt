\chapter*{GENERAL DESCRIPTION OF WORK}
%\addcontentsline{toc}{chapter}{Introduction}

Master's thesis, 55 pp., 13 sources, 17 images.
CRYPTOGRAPHY, NEURAL CRYPTOGRAPHY, NEURAL NETWORKS, CONVERSION OF FAYSTEL.
\bigskip

The object of study is the cryptographic transformations of Feistel.
\bigskip

The purpose of the work "--- the construction of reliability estimates of cryptographic transformations of Feistel using artificial neural networks.
Also, the aim of the work is to build artificial neural networks approximating the Feistel transform.
\bigskip

Research Methods "--- building neural networks, studying the results of computer experiments, working with scientific materials.
\bigskip

Results "--- mathematical models of cryptographic primitives and Feistel transforms were defined. Direct distribution neural networks with single-layer and multilayer perceptrons were built. Also in the process, a software package was generated that generated model data of the Feistel cryptographic transform and cryptographic primitives.
\bigskip

As a result of the work, it was found that an artificial neural network can approximate cryptographic primitives with high accuracy. 
Neural networks with multilayer perceptrons are able to approximate the Feistel transforms.

\bigskip
The results can be applied in cryptanalysis of cryptosystems based on Feistel networks.