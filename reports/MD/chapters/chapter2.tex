\chapter{Аппроксимация криптографических примитивов преобразования Фейстеля}
\label{c:chapter2}
\section{Математическое описание криптографических примитивов}

В данной главе проведем исследования, в которых мы попытаемся аппроксимировать криптографические примитивы преобразования Фейстеля.
В дальнейшем полученные результаты будут использованы в Главе \ref{c:chapter3}.

\bigskip
Рассмотрим частный случай сети Фейстеля ГОСТ 28147-89. Опишем однотактовое преобразование шифрования ГОСТ 28147-89:
\bigskip

$Y = g(X, K) = g(X_1 || X_2, K) \equiv (S[X_1 \boxplus K] \ll 11) \oplus X_2 || X_1,$ где

$X \in V_{64}$ - вектор входных данных,

$Y \in V_{64}$ - выходные данные,

$K \in V_{32}$ - ключ,

$\ll 11$ - циклический сдвиг влево на 11 бит,

$S$ - стандартный S-блок из ГОСТ-28147.


\bigskip
Определим следующие математические модели, для которых в дальнейшем будут построены нейронные сети и проведены компьютерные эксперименты.
\bigskip
\begin{enumerate}
	\item
	$Y_{g_0} = g_0(x) = g_0(x_1 || x_2)\equiv x_1 \oplus x_2,$ где
	
	$x \in V_{8}$ - вектор входных данных,
	
	$x_1, x_2 \in V_{4}$ - левая и правая часть входного вектора,
	
	$Y_{g_0} \in V_{4}$ - выходные данные модели $g_0$;
	\bigskip
	
	\item
	$Y_{g_1} = g_1(x) = g_1(x_1 || x_2, k) \equiv S[x_1] \oplus x_2,$ где
	
	$x \in V_{8}$ - вектор входных данных,
	
	$x_1, x_2 \in V{4}$ - левая и правая часть входного вектора,
	
	$Y_{g_1} \in V_{4}$ - выходные данные модели $g_1$,
	
	$S$ - первый узел из стандартного S-блока из ГОСТ-28147 
	
	($ S = \{13, 2, 8, 4, 6, 15, 11, 1, 10, 9, 3, 14, 5, 0, 12, 7\}$);
	\bigskip
	
	\item
	$Y_{g_2} = g_2(x) = g_2(x_1 || x_2, k) \equiv S[x_1 \boxplus k] \oplus x_2,$ где
	
	$x \in V_{8}$ - вектор входных данных,
	
	$x_1, x_2 \in V{4}$ - левая и правая часть входного вектора,
	
	$k \in V_{4}$ - некоторый неизвестный постоянный в эксперименте ключ,
	
	$Y_{g_2} \in V_{4}$ - выходные данные модели $g_2$,
	
	$S$ - первый узел из стандартного S-блока из ГОСТ-28147;
	
	%\item 
	%$Y_{g_3} = g_3(X) = g_3(x_1 || x_2, k) \equiv (S[x_1 \boxplus k] \ll 11) \oplus x_2,$ где
	%
	%$x \in V_{64}$ - вектор входных данных,
	%
	%$k \in V_{32}$ - некоторый неизвестный постоянный в эксперименте ключ,
	%	
	%$Y_{g_3} \in V_{32}$ - выходные данные модели $g_3$,
	%
	%$S$ - стандартный S-блок из ГОСТ-28147;
	%
	\item
	$Y_{g_4} = g_4(x) \equiv x \boxplus K,$ где
	
	$x \in V_{4}$ - вектор входных данных,
	
	$k \in V_{4}$ - некоторый неизвестный постоянный в эксперименте ключ,
	
	$Y_{g_4} \in V_{4}$ - выходные данные модели $g_4$;
	\bigskip
	
	\item
	$Y_{g_8} = g_8(x) \equiv x \boxplus K,$ где
	
	$x \in V_{8}$ - вектор входных данных,
	
	$k \in V_{8}$ - некоторый неизвестный постоянный в эксперименте ключ,
	
	$Y_{g_8} \in V_{8}$ - выходные данные модели $g_8$;
	\bigskip
	
	\item
	$Y_{g_{16}} = g_{16}(x) \equiv x \boxplus K,$ где
	
	$x \in V_{16}$ - вектор входных данных,
	
	$k \in V_{16}$ - некоторый неизвестный постоянный в эксперименте ключ,
	
	$Y_{g_{16}} \in V_{16}$ - выходные данные модели $g_{16}$;
	\bigskip
	
	\item
	$Y_{g_{32}} = g_{32}(x) \equiv x \boxplus K,$ где
	
	$x \in V_{32}$ - вектор входных данных,
	
	$k \in V_{32}$ - некоторый неизвестный постоянный в эксперименте ключ,
	
	$Y_{g_{32}} \in V_{g_{32}}$ - выходные данные модели $g_{32}$.
	\bigskip
	
\end{enumerate}

\bigskip
Опишем определённые ранее модели как частный случай однотактового преобразования шифрования ГОСТ 28147-89:
\begin{enumerate}
	\item Модель $g_0$ является однотактовым преобразованием шифрования ГОСТ 28147-89, при следующих условиях:
	\begin{itemize}
		\item S-блок - прямая таблица подстановки (подстановка при которой исходное значение переходит в такое же значение).
		\item $K = 0^{32}$.
		\item Отсутствует сдвиг влево на 11 бит.
		\item Рассматриваются только первые 4-бита $X_1, X_2, Y$.
	\end{itemize}
	\item Модель $g_1$ является однотактовым преобразованием шифрования ГОСТ 28147-89, при следующих условиях:
	\begin{itemize}
		\item S-блок - стандартный S-блок из ГОСТ-28147.
		\item $K = 0^{32}$.
		\item Отсутствует сдвиг влево на 11 бит.
		\item Рассматриваются только первые 4-бита $X_1, X_2, Y$.
	\end{itemize}
	\item Модель $g_2$ является однотактовым преобразованием шифрования ГОСТ 28147-89, при следующих условиях:
	\begin{itemize}
		\item S-блок - стандартный S-блок из ГОСТ-28147.
		\item $K$ - некоторый неизвестный постоянный в эксперименте ключ.
		\item Отсутствует сдвиг влево на 11 бит.
		\item Рассматриваются только первые 4-бита $X_1, X_2, Y$.
	\end{itemize}
	%\item Модель $g_3$ является однотактовым преобразованием шифрования ГОСТ 28147-89, при следующих условиях:
	%\begin{itemize}
	%	\item Рассматриваются только правая половина выходного ветора $Y$.
	%\end{itemize}
	\item Модель $g_4$ является однотактовым преобразованием шифрования ГОСТ 28147-89, при следующих условиях:
	\begin{itemize}
		\item S-блок - прямая таблица подстановки (подстановка при которой исходное значение переходит в такое же значение).
		\item $K$ - некоторый неизвестный постоянный в эксперименте ключ.
		\item Отсутствует сдвиг влево на 11 бит.
		\item Левая часть входного вектора отсутствует, либо равна $0^{32}$.
		\item Рассматриваются только первые 4-бита $X, Y, K$.
	\end{itemize}
	\item Модель $g_8$ является однотактовым преобразованием шифрования ГОСТ 28147-89, при следующих условиях:
	\begin{itemize}
		\item S-блок - прямая таблица подстановки (подстановка при которой исходное значение переходит в такое же значение).
		\item $K$ - некоторый неизвестный постоянный в эксперименте ключ.
		\item Отсутствует сдвиг влево на 11 бит.
		\item Левая часть входного вектора отсутствует, либо равна $0^{32}$.
		\item Рассматриваются только первые 8-бита $X, Y, K$.
\end{itemize}
	\item Модель $g_{16}$ является однотактовым преобразованием шифрования ГОСТ 28147-89, при следующих условиях:
\begin{itemize}
	\item S-блок - прямая таблица подстановки (подстановка при которой исходное значение переходит в такое же значение).
	\item $K$ - некоторый неизвестный постоянный в эксперименте ключ.
	\item Отсутствует сдвиг влево на 11 бит.
	\item Левая часть входного вектора отсутствует, либо равна $0^{32}$.
	\item Рассматриваются только первые 16-бита $X, Y, K$.
\end{itemize}
	\item Модель $g_{32}$ является однотактовым преобразованием шифрования ГОСТ 28147-89, при следующих условиях:
\begin{itemize}
	\item S-блок - прямая таблица подстановки (подстановка при которой исходное значение переходит в такое же значение).
	\item $K$ - некоторый неизвестный постоянный в эксперименте ключ.
	\item Отсутствует сдвиг влево на 11 бит.
	\item Левая часть входного вектора отсутствует, либо равна $0^{32}$.
	\item Рассматриваются только первые 32-бита $X, Y, K$.
\end{itemize}
\end{enumerate}

\newpage

\section{Аппроксимация криптографических примитивов с помощью нейронных сетей}

\subsection{Описание условий компьютерного эксперимента и используемых нейронных сетей}
Основываясь на предположения из главы \ref{c:chapter2},
чтобы аппроксимировать определённые ранее модели, необходимо построить искусственную нейронную сеть, с одним скрытым слоем.
\bigskip

Чтобы подтвердить наши теоретические предположения, построим следующие нейронные сети:
\begin{enumerate}
	\item однослойную нейронную сеть,
	\item многослойная нейронная сеть с одним скрытым слоем, с переменным количеством нейронов на скрытом слое.
\end{enumerate}
\bigskip

Компьютерные эксперименты для определённых ранее моделей проведем по следующему плану:
\begin{enumerate}
	\item Генерируем модельные данные с помощью разработанного генератора (обучающую и экзаменационную выборку).
	\item Строим однослойную нейронной сети, оцениваем точность данной нейронной сети.
	\item Если точность равна 100\%, принимаем решение что однослойная нейронная сеть аппроксимирует данную модель.
	\item В обратном случае строим график зависимости точности построенной нейронной сети с одним скрытым слоем от количества нейронов на скрытом слое.
	\item Анализируем полученный график и находим нейронную сеть с минимальным количеством нейронов на скрытом слое, точность аппроксимации которой равна или близка 100\%.
	\item Для найденной нейронной сети строим график зависимости точность от количества итераций обучения.
\end{enumerate}

\bigskip
Для оценки точность построенной нейронной сети использовалось расстояние Хэмминга:

\begin{equation}
w(y, \hat{y}) = \sum_{i=1}^{j} y_i \oplus \hat{y}_i,\ y_i \in V_j, V \in \{0,1\}.
\end{equation}
\bigskip

Для оценки точности проведенного численного эксперимента использовалось следующая функция:

\begin{equation}
\hat{f} =L - \dfrac{1}{T_e}\sum_{j=1}^{T_e}w(y^{(j)}, \hat{y^{(j)}}),
\end{equation}
где $L$ - количество бит выходного вектора оцениваемой модели,\\
$T_e$ - размер тестовой (экзаменационной) выборки,\\
$\hat{y^{(j)}}$ - выходной вектор предсказанный нейронной сетью,\\
$y^{(j)}$ - эталонный выходной вектор.

\bigskip
В качестве функции потерь в процессе обучения нейронной сети использовалось среднеквадратичное отклонение (для многослойного персептрона) и кросс-энтропия (для однослойного персептрона).

\bigskip
В качестве функции активации использовалась сигмоидальная функция:
\begin{equation}
\phi(z) = \frac{1}{1 + e ^{-z}}.
\end{equation}

\bigskip
\newpage
\subsection{Описание данных эксперимента}

Компьютерные эксперименты проводились на следующих данных:
\begin{enumerate}
	\item Модель $g_0$ :
	\begin{itemize}
		\item обучающая выборка $T_o=128$ пар $(x, y)$;
		\item экзаменационная выборка $T_e=24$ пар $(x, y)$.
	\end{itemize}
	
	\item Модель $g_1$ :
	\begin{itemize}
		\item обучающая выборка $T_o=128$ пар $(x, y)$;
		\item экзаменационная выборка $T_e=24$ пар $(x, y)$.
	\end{itemize}
	
	\item Модель $g_2$ :
	\begin{itemize}
		\item обучающая выборка $T_o=128$ пар $(x, y)$;
		\item экзаменационная выборка $T_e=24$ пар $(x, y)$.
	\end{itemize}

	%\item Модель $G_3$ :
	%\begin{itemize}
	%	\item обучающая выборка $T_o=2048$ пар $(x, y)$;
	%	\item экзаменационная выборка $T_e=409$ пар $(x, y)$.
	%\end{itemize}
	
	\item Модель $g_4$ :
	\begin{itemize}
		\item обучающая выборка $T_o=10$ пар $(x, y)$;
		\item экзаменационная выборка $T_e=5$ пар $(x, y)$.
	\end{itemize}
	
	\item Модель $g_8$:
	\begin{itemize}
		\item обучающая выборка $T_o=128$ пар $(x, y)$;
		\item экзаменационная выборка $T_e=24$ пар $(x, y)$.
	\end{itemize}
	
	\item Модель $g_{16}$:
	\begin{itemize}
		\item обучающая выборка $T_o=512$ пар $(x, y)$;
		\item экзаменационная выборка $T_e=102$ пар $(x, y)$.
	\end{itemize}
	
	\item Модель $g_{32}$:
	\begin{itemize}
		\item обучающая выборка $T_o=2048$ пар $(x, y)$;
		\item экзаменационная выборка $T_e=409$ пар $(x, y)$.
	\end{itemize}
	
\end{enumerate}

\newpage
\section{Численные результаты}
\subsection{Модель $g_0$}

\begin{figure}[H]
	\includegraphics[width=0.8\linewidth]{results/g0_0.png}
	
	\caption{График точности построенной однослойной нейронной сети модели $g_0$ от количества итераций обучения}
	
\end{figure}

\begin{figure}[H]
	\includegraphics[width=0.8\linewidth]{results/g0_1.png}
	
	\caption{График точности построенной нейронной сети с одним скрытым слоем модели $g_0$ от количества нейронов на скрытом слое}
	\label{graphic0}
\end{figure}

Из графика \ref{graphic0} видно, что нейронная сеть с одним скрытым слоем, с 8 нейронами на скрытом слое, показывает необходимую точность. Поэтому построим график зависимости точности данной нейронной сети от количества итераций обучения.

\begin{figure}[H]
	\includegraphics[width=0.8\linewidth]{results/g0_1_x.png}
	
	\caption{График точности построенной нейронной сети с одним скрытым слоем модели $g_0$  от количества итераций обучения}
\end{figure}

\subsection{Модель $g_1$}
\begin{figure}[H]
	\includegraphics[width=0.8\linewidth]{results/g1_0.png}
	
	\caption{График точности построенной однослойной нейронной сети модели $g_1$ от количества итераций обучения}
	
\end{figure}

\begin{figure}[H]
	\includegraphics[width=0.8\linewidth]{results/g1_1.png}
	
	\caption{График точности построенной нейронной сети с одним скрытым слоем модели $g_1$ от количества нейронов на скрытом слое}
	\label{graphic1}
\end{figure}

Из графика \ref{graphic1} видно, что нейронная сеть с одним скрытым слоем, с 8 нейронами на скрытом слое, показывает необходимую точность. Поэтому построим график зависимости точности данной нейронной сети от количества итераций обучения.

\begin{figure}[H]
	\includegraphics[width=0.8\linewidth]{results/g1_1_x.png}
	
	\caption{График точности построенной нейронной сети с одним скрытым слоем модели $g_1$ от количества нейронов на скрытом слое}
\end{figure}

\subsection{Модель $g_2$}
\begin{figure}[H]
	\includegraphics[width=0.8\linewidth]{results/g2_0.png}
	
	\caption{График точности построенной однослойной нейронной сети модели $g_2$ от количества итераций обучения}
	
\end{figure}


\begin{figure}[H]
	\includegraphics[width=0.8\linewidth]{results/g2_1.png}
	
	\caption{График точности построенной нейронной сети с одним скрытым слоем модели $g_2$ от количества нейронов на скрытом слое}
	\label{graphic2}
\end{figure}

Из графика \ref{graphic2} видно, что нейронная сеть с одним скрытым слоем, с 8 нейронами на скрытом слое, показывает необходимую точность. Поэтому построим график зависимости точности данной нейронной сети от количества итераций обучения.

\begin{figure}[H]
	\includegraphics[width=0.8\linewidth]{results/g2_1_x.png}
	
	\caption{График точности построенной нейронной сети с одним скрытым слоем модели $g_0$ от количества нейронов на скрытом слое}
\end{figure}


%\subsection{Модель $g_3$}
%\begin{figure}[H]
%	\includegraphics[width=0.8\linewidth]{results/g2_0.png}
%	
%	\caption{График точности построенной однослойной нейронной сети модели $g_2$ от количества итераций обучения}
%	
%\end{figure}
%
%
%\begin{figure}[H]
%	\includegraphics[width=0.8\linewidth]{results/g2_1.png}
%	
%	\caption{График точности построенной нейронной сети с одним скрытым слоем модели $g_2$ от количества нейронов на скрытом слое}
%\end{figure}
%
%Из предыдущего графика видно, что нейронная сеть с одним скрытым слоем, с 8 нейронами на скрытом слое, показывает высокий результат.
%Поэтому построим график зависимости точности данной сети от количества итераций обучения.
%
%\begin{figure}[H]
%	\includegraphics[width=0.8\linewidth]{results/g2_1_x.png}
%	
%	\caption{График точности построенной нейронной сети с одним скрытым слоем модели $g_0$ от количества нейронов на скрытом слое}
%\end{figure}


\subsection{Модель $g_4$}
\begin{figure}[H]
	\includegraphics[width=0.8\linewidth]{results/g4_0.png}
	
	\caption{График точности построенной однослойной нейронной сети модели $g_4$ от количества итераций обучения}
	
\end{figure}


\begin{figure}[H]
	\includegraphics[width=0.8\linewidth]{results/g4_1.png}
	
	\caption{График точности построенной нейронной сети с одним скрытым слоем модели $g_4$ от количества нейронов на скрытом слое}
	\label{graphic4}
\end{figure}

Из графика \ref{graphic4} видно, что нейронная сеть с одним скрытым слоем, с 4 нейронами на скрытом слое, показывает необходимую точность. Поэтому построим график зависимости точности данной нейронной сети от количества итераций обучения.я.

\begin{figure}[H]
	\includegraphics[width=0.8\linewidth]{results/g4_1_x.png}
	
	\caption{График точности построенной нейронной сети с одним скрытым слоем модели $g_4$ от количества нейронов на скрытом слое}
\end{figure}

\subsection{Модель $g_8$}
\begin{figure}[H]
	\includegraphics[width=0.8\linewidth]{results/g8_0.png}
	
	\caption{График точности построенной однослойной нейронной сети модели $g_8$ от количества итераций обучения}
	
\end{figure}


\begin{figure}[H]
	\includegraphics[width=0.8\linewidth]{results/g8_1.png}
	
	\caption{График точности построенной нейронной сети с одним скрытым слоем модели $g_8$ от количества нейронов на скрытом слое}
	\label{graphic8}
\end{figure}

Из графика \ref{graphic8} видно, что нейронная сеть с одним скрытым слоем, с 8 нейронами на скрытом слое, показывает необходимую точность. Поэтому построим график зависимости точности данной нейронной сети от количества итераций обучения.

\begin{figure}[H]
	\includegraphics[width=0.8\linewidth]{results/g8_1_x.png}
	
	\caption{График точности построенной нейронной сети с одним скрытым слоем модели $g_8$ от количества нейронов на скрытом слое}
\end{figure}

\subsection{Модель $g_{16}$}
\begin{figure}[H]
	\includegraphics[width=0.8\linewidth]{results/g16_0.png}
	
	\caption{График точности построенной однослойной нейронной сети модели $g_{16}$ от количества итераций обучения}
	
\end{figure}


\begin{figure}[H]
	\includegraphics[width=0.8\linewidth]{results/g16_1.png}
	
	\caption{График точности построенной нейронной сети с одним скрытым слоем модели $g_{16}$ от количества нейронов на скрытом слое}
	\label{graphic16}
\end{figure}

Из графика \ref{graphic16} видно, что нейронная сеть с одним скрытым слоем, с 16 нейронами на скрытом слое, показывает необходимую точность. Поэтому построим график зависимости точности данной нейронной сети от количества итераций обучения.

\begin{figure}[H]
	\includegraphics[width=0.8\linewidth]{results/g16_1_x.png}
	
	\caption{График точности построенной нейронной сети с одним скрытым слоем модели $g_{16}$ от количества нейронов на скрытом слое}
\end{figure}

\subsection{Модель $g_{32}$}
\begin{figure}[H]
	\includegraphics[width=0.8\linewidth]{results/g32_0.png}
	
	\caption{График точности построенной однослойной нейронной сети модели $g_{32}$ от количества итераций обучения}
	
\end{figure}


\begin{figure}[H]
	\includegraphics[width=0.8\linewidth]{results/g32_1.png}
	
	\caption{График точности построенной нейронной сети с одним скрытым слоем модели $g_{32}$ от количества нейронов на скрытом слое}
	\label{graphic32}
\end{figure}

Из графика \ref{graphic32} видно, что нейронная сеть с одним скрытым слоем, с 64 нейронами на скрытом слое, показывает необходимую точность. Поэтому построим график зависимости точности данной нейронной сети от количества итераций обучения.

\begin{figure}[H]
	\includegraphics[width=0.8\linewidth]{results/g32_1_x.png}
	
	\caption{График точности построенной нейронной сети с одним скрытым слоем модели $g_{32}$ от количества нейронов на скрытом слое}
\end{figure}

\subsection{Обобщение полученных результатов}
Основываясь на проведенных экспериментах построим таблицу в которой сравним модели и построенный нейронные сети с минимальным количеством параметров.
В таблице представим следующие параметры сравнения:
размер обучающей выборки, размер тестовой выборки, параметры построенной нейронной сети, точность нейронной сети, потери при обучении, время затраченное на обучение.

\bigskip

Введем следующие обозначения:
\begin{itemize}
	\item $N$ - количество нейронов на скрытом слое,
	\item $NP$ - число обучаемых параметров cети, 
	\item $T_o$ - размер обучающей выборки,
	\item $T_e$ - размер тестовой (экзаменационной) выборки,
	\item $\hat{f}$ - точность построенной нейронной сети,
	\item $l$ - потери при обучении,
	\item $\Psi$ - время обучения нейронной сети в секундах,
	\item $\Sigma$ - количество эпох обучения.
\end{itemize}

\begin{table}[H]
	\begin{tabular}{|l|l|l|l|l|l|l|l|l|l|}
		\hline
		Модель    & НС   & $N$&$NP$& $T_o$       & $T_e$  & $\hat{f}$ & l   & $\Sigma$ & $\Psi$ \\ \hline
		$ g_{0} $ & NN 	 & 0  & 32 & 128 & 24  & 3.2000 & 0.5927 & 4000 & 1.1266 \\ \hline
		$ g_{0} $ & MNN  & 8  & 96 & 128 & 24  & 4.0000 & 0.0631 & 5000 & 1.7937 \\ \hline
		$ g_{1} $ & NN 	 & 0  & 32 & 128 & 24  & 2.3500 & 0.6519 & 22000 & 7.2215 \\ \hline
		$ g_{1} $ & MNN  & 8  & 96 & 128 & 24  & 4.0000 & 0.1136 & 20000 & 7.1536 \\ \hline
		$ g_{2} $ & NN 	 & 0  & 32 & 128 & 24  & 2.6923 & 0.6338 & 14000 & 6.4174 \\ \hline
		$ g_{2} $ & MNN  & 8  & 96 & 128 & 24  & 4.0000 & 0.1547 & 25000 & 9.3551 \\ \hline
		%$g_{3}$ & MNN   &             &                   &         2048&   409  &           &     & 10000    &       \\ \hline
		%$g_{3}$ & NN    &             &                   &         2048&   409  &           &     & 10000    &       \\ \hline
		$ g_{4} $ & NN 	 & 0  & 16 & 10  & 5   & 3.5000 & 0.6237 & 2000 & 3.6132 \\ \hline
		$ g_{4} $ & MNN  & 4  & 32 & 10  & 5   & 4.0000 & 0.0918 & 5000 & 4.7954 \\ \hline
		$ g_{8} $ & NN 	 & 0  & 64 & 128 & 24  & 7.5385 & 0.5015 & 12000 & 8.2843 \\ \hline
		$ g_{8} $ & MNN  & 8  & 128 & 128& 24  & 8.0000 & 0.0603 & 5000 & 5.3191 \\ \hline
		$ g_{16} $ & NN  & 0  & 256 & 512& 129 & 14.5577 & 0.5335 & 26000 & 16.2864 \\ \hline
		$ g_{16} $ & MNN & 16 & 512 & 512& 129 & 16.0000 & 0.0469 & 15000 & 10.5211 \\ \hline
		$ g_{32} $ & NN  & 0  & 1024 & 2048 & 409 & 25.5520 & 0.1447 & 26000 & 21.0100 \\ \hline
		$ g_{32} $ & MNN & 64 & 4096 & 2048 & 409 & 32.0000 & 0.0211 & 10000 & 13.2120 \\ \hline
		
	\hline
	\end{tabular}
	\caption{Сравнение построенных нейронных сетей}
	\label{diff_g}
\end{table}
\bigskip

Из представленных выше результатов (графики и таблица \ref{diff_g}), можно сделать вывод, 
что точность нейронной сети с одним скрытым слоем значительно выше точности обычной нейронной сети прямого распространения.
Следовательно, наши теоретические предположения были верны.


