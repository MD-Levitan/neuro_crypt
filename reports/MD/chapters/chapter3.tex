  \chapter{Оценка надежности криптографического преобразования Фейстеля с помощью его аппроксимации нейронной сетью}
  \label{c:chapter3}
  \section{Описание используемой модели криптографического преобразования Фейстеля}
  В данной главе рассмотрим оценку надежности криптографического преобразования Фейстеля с помощью его аппроксимации нейронной сетью.
  \bigskip
  
  Для оценки стойкости итерационного проебразования определим согласно \cite{crypto} модельное однотактовое криптографическое преобразования Фейстеля. 
  Примем следующие обозначения:
  \begin{itemize}
  	\item  $V = {0, 1};$
  	\item  $I$ - количество тактов;
  	\item  $N$ - размерность преобразования Фейстеля;
  	\item  $X = (x_i) \in V^{2N}, X = (X_1 \| X_2) \in V^{2N}, X_i \in V^N, i=1,2;$
  	\item  $Y = (y_i) \in V^{2N}, Y = (Y_1 \| Y_2) \in V^{2N}, Y_i \in V^N, i=1,2;$
  	\item  $<X>, <Y> \in \{0,1,...,2^{2N}-1\}$ - числовое представление векторов $X$, $Y$;
  	\item  $K=(K1\|...\|K_8)=(k_1,...,k_{8N})$ - ключ тактового преобразования;
  	\item  $K_i=(k_{(i-1)N+1},k_{(i-1)N+2},...,k_{iN}) \in V^{N}, i=1,...,8$ - $i$-ый подключ преобразования;
  	\item  $<K_i> \in \{0,1,...,2^{N}-1\}$ - числовое представление двоичного вектора $K_i$;
  	\item  $\lll L$ - циклический сдвиг влево на $L$ бит;
  	\item   $f(\cdot):V^N \times V^N \rightarrow V^N$ - функция криптографического преобразования Фейстеля;
  	\item $g(\cdot):V^N \rightarrow V^N$ - расписание ключей.
  \end{itemize}

\tikzset{%
	block/.style    = {draw, thick, rectangle, minimum height = 3em,
		minimum width = 3em},
	sum/.style      = {draw, circle, node distance = 2cm}, % Adder
	input/.style    = {coordinate}, % Input
	output/.style   = {coordinate} % Output
}
% Defining string as labels of certain blocks.
\newcommand{\suma}{\Large$+$}
\newcommand{\derv}{\huge$\frac{d}{dt}$}

\begin{figure}[H]
	\begin{tikzpicture}[auto, thick, node distance=2cm, >=triangle 45]
	\draw
	% Drawing the blocks of first filter :
	node [block] (x1) {\Large$X_1$}
	node at (10,0)[block] (x2) {\Large$X_2$}
	
	node at (10,-3)[sum] (f) {\Large$f(\cdot)$}
	node at (13,-3)[sum] (k) {\Large$K_i$}
	
	node at (10,-6)[block] (l) {\Large$\lll L$}
	node at (10,-9)[sum] (sum) {\Large$\oplus$}
	
	node at (0,-12) [block] (y1) {\Large$Y_1$}
	node at (10,-12)[block] (y2) {\Large$Y_2$};
	% Joining blocks. 
	
	\draw[->](x1) -- node {} (sum);
	\draw[->](x2) -- node {} (f);
	\draw[->](x2) -- node {} (y1);
	\draw[->](k) -- node {} (f);
	\draw[->](f) -- node {} (l);
	\draw[->](l) -- node {} (sum);
	\draw[->](sum) -- node {} (y2);
	
	
	\end{tikzpicture}
	\caption{Модельное 1-тактовое преобразование с $i$-ым подключом}
\end{figure}



  \bigskip
  В данном исследовании будем использовать использовать следующию предположения:
  \begin{enumerate}
  \item Размерность преобразования Фейстеля:
  
  		 $N = 8$.
  \item Функция криптографического преобразования Фейстеля:	
  
  		$f(X_2;K) = S[X_2 \boxplus K]$, где $S$ - два первых стандартных S-блока из ГОСТ-28147.
  
  \item Расписание ключей:
  
        $g(K)=K_1, ..., K_8; K_1,...K_8;...;K_1,...K_8$.
  
  \item Параметры $L$, $I$ - изменяемые параметры в следующих диапозонах:
  \begin{itemize}
  	\item $L \in \{0,1,...,7\},$
  	
  	\item $I \in \{1,2,...,8\}.$
  \end{itemize}
  \end{enumerate}
  
  \bigskip
  Определим полученные математические модели:
  
  $f_{I-L}$ "--- преобразования Фейстеля,
  
  где $I \in \{1,2,...,8\}$ "--- количество тактов,
  
  $L \in \{0,1,...,7\}$ "--- количество битов при циклическом сдвиге влево.  	
  \bigskip	
  
  \newpage
  \section{Описание используемых нейронных сетей}
  \subsection{Описание условий компьютерного эксперимента и используемых нейронных сетей}
  
Для решения поставленных задач использовались следующие нейронные сети:
 
\begin{enumerate}
 	\item однослойная нейронная сеть,
	\item многослойная нейронная сеть с одним скрытым слоем, с переменным количеством нейронов на скрытом слое,
	\item многослойная нейронная сеть с двумя скрытым слоем, с переменным количеством нейронов на скрытом слое,
	\item многослойная нейронная сеть с тремя скрытым слоем, с переменным количеством нейронов на скрытом слое.
	%\item Многослойная нейронная сеть с четырмя скрытым слоем, с переменным количеством нейронов на скрытом слое;
\end{enumerate}
  
  %Численные эксперименты для определённых ранее моделей проведем по следующему плану:
  %\begin{enumerate}
  %	\item Генерируем модельные данные с помощью разработанного генератора (обучающую и экзаменационную выборку).
  %	\item Строим описанные выше нейронные сети для однотактового криптографического преобразования Фейстеля.
  %	\item Оценим зависимость точности от чилса битов в циклическом сдвиге влева.
  	
  %\end{enumerate}
  
  \bigskip
  Для оценки точность построенной нейронной сети использовалось расстояние Хэмминга:
  
  \begin{equation}
  w(y, \hat{y}) = \sum_{i=1}^{j} y_i \oplus \hat{y}_i,\ y_i \in V_j, V \in \{0,1\}.
  \end{equation}
  \bigskip
  
  Для оценки точности проведенного компьютерного эксперимента использовалось следующая функция:
  
  \begin{equation}
  \hat{f} =L - \dfrac{1}{T_e}\sum_{j=1}^{T_e}w(y^{(j)}, \hat{y^{(j)}}),
  \end{equation}
  где $L$ - количество бит выходного вектора оцениваемой модели,\\
  $T_e$ - размер тестовой (экзаменационной) выборки,\\
  $\hat{y^{(j)}}$ - выходной вектор предсказанный нейронной сетью,\\
  $y^{(j)}$ - эталонный выходной вектор.
  
  \bigskip
  В качестве функции потерь в процессе обучения нейронной сети использовалось среднеквадратичное отклонение.
  
  \bigskip
  В качестве функции активации использовалась сигмоидальная функция:
  \begin{equation}
  \phi(z) = \frac{1}{1 + e ^{-z}}.
  \end{equation}
  
  \bigskip
  \newpage
  \subsection{Описание данных эксперимента}
  
  Компьютерные эксперименты проводились на следующих данных:
  \begin{itemize}
  		\item обучающая выборка $T_o=10240$ пар $(x, y)$;
  		\item экзаменационная выборка $T_e=1024$ пар $(x, y)$.
  \end{itemize}
  
  
  \newpage 
  \section{Численные результаты}
  \subsection{Модели $f_{1}$}
Рассмотрим модели преобразования Фестеля с одним тактом при изменяющемся параметре $L$.
Построим однослойную нейронную сеть и нейронную сеть с одним скрытым слоем, с 32 нейронами на данном слое.
Для данных моделей и нейронных сетей построим графики точности, и сравним полученные результаты.

    \bigskip
  \begin{figure}[H]
  	\includegraphics[width=0.7\linewidth]{results/f1-0_0.png}
  	
  	\caption{График точности построенной однослойной нейронной сети модели $f_{1-0}$ от количества итераций обучения}
  	
  \end{figure}
  
  \begin{figure}[H]
  	\includegraphics[width=0.7\linewidth]{results/f1-0_1_x.png}
  	
  	\caption{График точности построенной нейронной сети с одним скрытым слоем модели $f_{1-0}$  от количества итераций обучения}
  \end{figure}
  
    \begin{figure}[H]
  	\includegraphics[width=0.7\linewidth]{results/f1-1_0.png}
  	
  	\caption{График точности построенной однослойной нейронной сети модели $f_{1-1}$ от количества итераций обучения}
  	
  \end{figure}
  
  \begin{figure}[H]
  	\includegraphics[width=0.7\linewidth]{results/f1-1_1_x.png}
  	
  	\caption{График точности построенной нейронной сети с одним скрытым слоем модели $f_{1-1}$  от количества итераций обучения}
  \end{figure}

  \begin{figure}[H]
	\includegraphics[width=0.7\linewidth]{results/f1-2_0.png}
	
	\caption{График точности построенной однослойной нейронной сети модели $f_{1-2}$ от количества итераций обучения}
	
	\end{figure}

	\begin{figure}[H]
		\includegraphics[width=0.7\linewidth]{results/f1-2_1_x.png}
	
	\caption{График точности построенной нейронной сети с одним скрытым слоем модели $f_{1-2}$  от количества итераций обучения}
	\end{figure}
  
    \begin{figure}[H]
  	\includegraphics[width=0.7\linewidth]{results/f1-3_0.png}
  	
  	\caption{График точности построенной однослойной нейронной сети модели $f_{1-3}$ от количества итераций обучения}
  	
  \end{figure}
  
  \begin{figure}[H]
  	\includegraphics[width=0.7\linewidth]{results/f1-3_1_x.png}
  	
  	\caption{График точности построенной нейронной сети с одним скрытым слоем модели $f_{1-3}$  от количества итераций обучения}
  \end{figure}
  
      \begin{figure}[H]
  	\includegraphics[width=0.7\linewidth]{results/f1-4_0.png}
  	
  	\caption{График точности построенной однослойной нейронной сети модели $f_{1-4}$ от количества итераций обучения}
  	
  \end{figure}
  
  \begin{figure}[H]
  	\includegraphics[width=0.7\linewidth]{results/f1-4_1_x.png}
  	
  	\caption{График точности построенной нейронной сети с одним скрытым слоем модели $f_{1-4}$  от количества итераций обучения}
  \end{figure}
  
      \begin{figure}[H]
  	\includegraphics[width=0.7\linewidth]{results/f1-5_0.png}
  	
  	\caption{График точности построенной однослойной нейронной сети модели $f_{1-5}$ от количества итераций обучения}
  	
  \end{figure}
  
  \begin{figure}[H]
  	\includegraphics[width=0.7\linewidth]{results/f1-5_1_x.png}
  	
  	\caption{График точности построенной нейронной сети с одним скрытым слоем модели $f_{1-5}$  от количества итераций обучения}
  \end{figure}
  
      \begin{figure}[H]
  	\includegraphics[width=0.7\linewidth]{results/f1-6_0.png}
  	
  	\caption{График точности построенной однослойной нейронной сети модели $f_{1-6}$ от количества итераций обучения}
  	
  \end{figure}
  
  \begin{figure}[H]
  	\includegraphics[width=0.7\linewidth]{results/f1-6_1_x.png}
  	
  	\caption{График точности построенной нейронной сети с одним скрытым слоем модели $f_{1-6}$  от количества итераций обучения}
  \end{figure}
  
      \begin{figure}[H]
  	\includegraphics[width=0.7\linewidth]{results/f1-7_0.png}
  	
  	\caption{График точности построенной однослойной нейронной сети модели $f_{1-7}$ от количества итераций обучения}
  	
  \end{figure}
  
  \begin{figure}[H]
  	\includegraphics[width=0.7\linewidth]{results/f1-7_1_x.png}
  	
  	\caption{График точности построенной нейронной сети с одним скрытым слоем модели $f_{1-7}$  от количества итераций обучения}
  \end{figure}
  
\begin{table}[H]
	\begin{tabular}{|l|l|l|l|l|l|l|l|l|l|}
		\hline
Модель   & НС  & $N$&$NP$&$T_o$& $T_e$  & $\hat{f}$ & l   & $\Sigma$ & $\Psi$ \\ \hline  
$ f1-0 $ & NN  & 0  & 256  & 5120 & 512 & 12.3248 & 0.1264 & 5000 & 3.5308 \\ \hline
$ f1-0 $ & MNN & 32 & 1024 & 5120 & 512 & 16.0000 & 0.0318 & 10000 & 11.2596 \\ \hline
$ f1-1 $ & NN  & 0  & 256  & 5120 & 512 & 12.6752 & 0.1258 & 10000 & 7.6633 \\ \hline
$ f1-1 $ & MNN & 32 & 1024 & 5120 & 512 & 16.0000 & 0.0141 & 10000 & 11.7257 \\ \hline
$ f1-2 $ & NN  & 0  & 256  & 5120 & 512 & 13.1679 & 0.1258 & 20000 & 17.4358 \\ \hline
$ f1-2 $ & MNN & 32 & 1024 & 5120 & 512 & 16.0000 & 0.0279 & 10000 & 13.2084 \\ \hline
$ f1-3 $ & NN  & 0  & 256  & 5120 & 512 & 13.1241 & 0.1230 & 20000 & 16.9254 \\ \hline
$ f1-3 $ & MNN & 32 & 1024 & 5120 & 512 & 16.0000 & 0.0338 & 10000 & 13.3615 \\ \hline
$ f1-4 $ & NN  & 0  & 256  & 5120 & 512 & 12.7847 & 0.1287 & 10000 & 11.4368 \\ \hline
$ f1-4 $ & MNN & 32 & 1024 & 5120 & 512 & 16.0000 & 0.0381 & 10000 & 14.1050 \\ \hline
$ f1-5 $ & NN  & 0  & 256  & 5120 & 512 & 12.6131 & 0.1168 & 25000 & 24.4405 \\ \hline
$ f1-5 $ & MNN & 32 & 1024 & 5120 & 512 & 16.0000 & 0.0320 & 10000 & 14.7981 \\ \hline
$ f1-6 $ & NN  & 0  & 256  & 5120 & 512 & 12.5912 & 0.1312 & 5000 & 8.7702 \\ \hline
$ f1-6 $ & MNN & 32 & 1024 & 5120 & 512 & 16.0000 & 0.0501 & 5000 & 10.5173 \\ \hline
$ f1-7 $ & NN  & 0  & 256  & 5120 & 512 & 12.7883 & 0.1165 & 20000 & 21.9568 \\ \hline
$ f1-7 $ & MNN & 32 & 1024 & 5120 & 512 & 16.0000 & 0.0194 & 10000 & 16.5015 \\ \hline
  
  \hline
\end{tabular}
\caption{Сравнение модели $f_1$ при различных $L$}
\label{diff_f1}
\end{table} 

Из представленных выше графиков и таблицы \ref{diff_f1}, можно сделать вывод что параметр $L$ не вносит существенных изменений в построение нейронной сети и точность аппроксимации. По этой причине, в дальнейшем исследовании (в рассмотрении следующих моделей) мы будем рассматривать только краевые случаи, а именно модели в которых $L \equiv 0$ и $L \equiv 8$. 

\subsection{Модели $f_{2}$}
  
  %\bigskip
  %\begin{figure}[H]
  %	\includegraphics[width=0.7\linewidth]{results/f2-0_0.png}
  %	
  %	\caption{График точности построенной однослойной нейронной сети модели $f_{2-0}$ от количества итераций обучения}
  %	
  %\end{figure}
  
  %\begin{figure}[H]
  %	\includegraphics[width=0.7\linewidth]{results/f2-0_1.png}
  %	
  %	\caption{График точности построенной нейронной сети с одним скрытым слоем модели $f_{2-0}$  от архитектуры перцептрона}
  %\end{figure}
  
  %\begin{figure}[H]
  %	\includegraphics[width=0.7\linewidth]{results/f2-0_1_x.png}
  %	
  %	\caption{График точности построенной однослойной нейронной сети модели $f_{1-1}$ от количества итераций обучения}
  %	
  %\end{figure}
  
  \begin{figure}[H]
  	\includegraphics[width=0.7\linewidth]{results/f2-0_2.png}
  	
  	\caption{График точности построенной многослойной нейронной сети модели $f_{2-0}$ от архитектуры перцептрона}
  	
  \end{figure}
  
  
  \begin{figure}[H]
  	\includegraphics[width=0.7\linewidth]{results/f2-0_3.png}
  	
  	\caption{График точности построенной многослойной нейронной сети модели $f_{2-0}$ от архитектуры перцептрона}	
  \end{figure}

	%Среди представленных нейронных сетей лучший результат показала нейронная сеть с двумя скрытыми слоями и 64 нейронами на обоих слоях.
	%Для данной нейронной сети построим график зависимости точности данной нейронной сети от количества итераций обучения.
	
   
   %\bigskip
  %\begin{figure}[H]
  %	\includegraphics[width=0.7\linewidth]{results/f2-7_0.png}
  %	
  %	\caption{График точности построенной однослойной нейронной сети модели $f_{2-7}$ от количества итераций обучения}
  %	
  %\end{figure}
  
  %\begin{figure}[H]
  %	\includegraphics[width=0.7\linewidth]{results/f2-7_1.png}
  %	
  %	\caption{График точности построенной нейронной сети с одним скрытым слоем модели $f_{2-7}$ от архитектуры перцептрона}
  %\end{figure}
  
  \begin{figure}[H]
  	\includegraphics[width=0.7\linewidth]{results/f2-7_2.png}
  	
  	\caption{График точности построенной многослойной нейронной сети модели $f_{2-7}$ от архитектуры перцептрона}
  	
  \end{figure}
  
  
  \begin{figure}[H]
  	\includegraphics[width=0.7\linewidth]{results/f2-7_3.png}
  	
  	\caption{График точности построенной многослойной нейронной сети модели $f_{2-7}$ от архитектуры перцептрона}	
  \end{figure}
  
	%Среди представленных нейронных сетей лучший результат показала нейронная сеть с двумя скрытыми слоями и 64 нейронами на обоих слоях.
%Для данной нейронной сети построим график зависимости точности данной нейронной сети от количества итераций обучения.
  
\subsection{Модели $f_{3}$}

\bigskip
\begin{figure}[H]
	\includegraphics[width=0.7\linewidth]{results/f3-0_2.png}
	
	\caption{График точности построенной многослойной нейронной сети модели $f_{3-0}$ от количества итераций обучения}
	
\end{figure}


\begin{figure}[H]
	\includegraphics[width=0.7\linewidth]{results/f3-0_3.png}
	
	\caption{График точности построенной многослойной нейронной сети модели $f_{3-0}$ от архитектуры перцептрона}	
\end{figure}

	%Среди представленных нейронных сетей лучший результат показала нейронная сеть с двумя скрытыми слоями и 64 нейронами на обоих слоях.
%Для данной нейронной сети построим график зависимости точности данной нейронной сети от количества итераций обучения.


\bigskip

\begin{figure}[H]
	\includegraphics[width=0.7\linewidth]{results/f3-7_2.png}
	
	\caption{График точности построенной многослойной нейронной сети модели $f_{3-7}$ от архитектуры перцептрона}
	
\end{figure}


\begin{figure}[H]
	\includegraphics[width=0.7\linewidth]{results/f3-7_3.png}
	
	\caption{График точности построенной многослойной нейронной сети модели $f_{3-7}$ от архитектуры перцептрона}	
\end{figure}

	%Среди представленных нейронных сетей лучший результат показала нейронная сеть с двумя скрытыми слоями и 64 нейронами на обоих слоях.
%Для данной нейронной сети построим график зависимости точности данной нейронной сети от количества итераций обучения.
  
\subsection{Модели $f_{4}$}

\begin{figure}[H]
	\includegraphics[width=0.7\linewidth]{results/f4-0_2.png}
	
	\caption{График точности построенной многослойной нейронной сети модели $f_{4-0}$ от архитектуры перцептрона}
	
\end{figure}


\begin{figure}[H]
	\includegraphics[width=0.7\linewidth]{results/f4-0_3.png}
	
	\caption{График точности построенной многослойной нейронной сети модели $f_{4-0}$ от архитектуры перцептрона}	
\end{figure}
	%Среди представленных нейронных сетей лучший результат показала нейронная сеть с двумя скрытыми слоями и 64 нейронами на обоих слоях.
%Для данной нейронной сети построим график зависимости точности данной нейронной сети от количества итераций обучения.


\bigskip


\begin{figure}[H]
	\includegraphics[width=0.7\linewidth]{results/f4-7_2.png}
	
	\caption{График точности построенной многослойной нейронной сети модели $f_{4-7}$ от архитектуры перцептрона}
	
\end{figure}


\begin{figure}[H]
	\includegraphics[width=0.7\linewidth]{results/f4-7_3.png}
	
	\caption{График точности построенной многослойной нейронной сети модели $f_{4-7}$ от архитектуры перцептрона}	
\end{figure}

	%Среди представленных нейронных сетей лучший результат показала нейронная сеть с двумя скрытыми слоями и 64 нейронами на обоих слоях.
%Для данной нейронной сети построим график зависимости точности данной нейронной сети от количества итераций обучения.

\subsection{Модели $f_{5}$}

\bigskip


\begin{figure}[H]
	\includegraphics[width=0.7\linewidth]{results/f5-0_2.png}
	
	\caption{График точности построенной многослойной нейронной сети модели $f_{5-0}$ от архитектуры перцептрона}
	
\end{figure}


\begin{figure}[H]
	\includegraphics[width=0.7\linewidth]{results/f5-0_3.png}
	
	\caption{График точности построенной многослойной нейронной сети модели $f_{5-0}$ от архитектуры перцептрона}
\end{figure}

	%Среди представленных нейронных сетей лучший результат показала нейронная сеть с двумя скрытыми слоями и 64 нейронами на обоих слоях.
%Для данной нейронной сети построим график зависимости точности данной нейронной сети от количества итераций обучения.


\bigskip


\begin{figure}[H]
	\includegraphics[width=0.7\linewidth]{results/f5-7_2.png}
	
	\caption{График точности построенной многослойной нейронной сети модели $f_{5-7}$ от архитектуры перцептрона}
	
\end{figure}


\begin{figure}[H]
	\includegraphics[width=0.7\linewidth]{results/f5-7_3.png}
	
	\caption{График точности построенной многослойной нейронной сети модели $f_{5-7}$ от архитектуры перцептрона}
\end{figure}

	%Среди представленных нейронных сетей лучший результат показала нейронная сеть с двумя скрытыми слоями и 64 нейронами на обоих слоях.
%Для данной нейронной сети построим график зависимости точности данной нейронной сети от количества итераций обучения.

\subsection{Модели $f_{6}$}

\bigskip


\begin{figure}[H]
	\includegraphics[width=0.7\linewidth]{results/f6-0_2.png}
	
	\caption{График точности построенной многослойной нейронной сети модели $f_{6-0}$ от архитектуры перцептрона}
	
\end{figure}


\begin{figure}[H]
	\includegraphics[width=0.7\linewidth]{results/f6-0_3.png}
	
	\caption{График точности построенной многослойной нейронной сети модели $f_{6-0}$ от архитектуры перцептрона}	
\end{figure}

	%Среди представленных нейронных сетей лучший результат показала нейронная сеть с двумя скрытыми слоями и 64 нейронами на обоих слоях.
%Для данной нейронной сети построим график зависимости точности данной нейронной сети от количества итераций обучения.


\bigskip


\begin{figure}[H]
	\includegraphics[width=0.7\linewidth]{results/f6-7_2.png}
	
	\caption{График точности построенной многослойной нейронной сети модели $f_{6-7}$ от архитектуры перцептрона}
	
\end{figure}


\begin{figure}[H]
	\includegraphics[width=0.7\linewidth]{results/f6-7_3.png}
	
\caption{График точности построенной многослойной нейронной сети модели $f_{6-7}$ от архитектуры перцептрона}
\end{figure}

	%Среди представленных нейронных сетей лучший результат показала нейронная сеть с двумя скрытыми слоями и 64 нейронами на обоих слоях.
%Для данной нейронной сети построим график зависимости точности данной нейронной сети от количества итераций обучения.

\subsection{Модели $f_{7}$}

\bigskip


\begin{figure}[H]
	\includegraphics[width=0.7\linewidth]{results/f7-0_2.png}
	
	\caption{График точности построенной многослойной нейронной сети модели $f_{7-0}$ от архитектуры перцептрона}
	
\end{figure}


\begin{figure}[H]
	\includegraphics[width=0.7\linewidth]{results/f7-0_3.png}
	
	\caption{График точности построенной многослойной нейронной сети модели $f_{7-0}$ от архитектуры перцептрона}
\end{figure}

	%Среди представленных нейронных сетей лучший результат показала нейронная сеть с двумя скрытыми слоями и 64 нейронами на обоих слоях.
%Для данной нейронной сети построим график зависимости точности данной нейронной сети от количества итераций обучения.


\bigskip


\begin{figure}[H]
	\includegraphics[width=0.7\linewidth]{results/f7-7_2.png}
	
	\caption{График точности построенной многослойной нейронной сети модели $f_{7-7}$ от архитектуры перцептрона}
	
\end{figure}


\begin{figure}[H]
	\includegraphics[width=0.7\linewidth]{results/f7-7_3.png}
	
	\caption{График точности построенной многослойной нейронной сети модели $f_{7-7}$ от архитектуры перцептрона}
\end{figure}

	%Среди представленных нейронных сетей лучший результат показала нейронная сеть с двумя скрытыми слоями и 64 нейронами на обоих слоях.
%Для данной нейронной сети построим график зависимости точности данной нейронной сети от количества итераций обучения.

\subsection{Модели $f_{8}$}

\bigskip


\begin{figure}[H]
	\includegraphics[width=0.7\linewidth]{results/f8-0_2.png}
	
	\caption{График точности построенной многослойной нейронной сети модели $f_{8-0}$ от архитектуры перцептрона}
	
\end{figure}


\begin{figure}[H]
	\includegraphics[width=0.7\linewidth]{results/f8-0_3.png}
	
	\caption{График точности построенной многослойной нейронной сети модели $f_{8-0}$ от архитектуры перцептрона}	
\end{figure}

	%Среди представленных нейронных сетей лучший результат показала нейронная сеть с двумя скрытыми слоями и 64 нейронами на обоих слоях.
%Для данной нейронной сети построим график зависимости точности данной нейронной сети от количества итераций обучения.


\bigskip


\begin{figure}[H]
	\includegraphics[width=0.7\linewidth]{results/f8-7_2.png}
	
	\caption{График точности построенной многослойной нейронной сети модели $f_{8-7}$ от архитектуры перцептрона}
	
\end{figure}


\begin{figure}[H]
	\includegraphics[width=0.7\linewidth]{results/f8-7_3.png}
	
	\caption{График точности построенной многослойной нейронной сети модели $f_{8-}$ от архитектуры перцептрона}	
\end{figure}

	%Среди представленных нейронных сетей лучший результат показала нейронная сеть с двумя скрытыми слоями и 64 нейронами на обоих слоях.
%Для данной нейронной сети построим график зависимости точности данной нейронной сети от количества итераций обучения.

  
\subsection{Обобщение полученных результатов}
Основываясь на проведенных экспериментах построим таблицу в которой сравним модели и построенный нейронные сети с минимальным количеством параметров.
В таблице представим следующие параметры сравнения:
размер обучающей выборки, размер тестовой выборки, параметры построенной нейронной сети, точность нейронной сети, потери при обучении, время затраченное на обучение.

\bigskip

Введем следующие обозначения:
\begin{itemize}
	\item $N$ - количество нейронов на скрытом слое,
	\item $NP$ - число обучаемых параметров cети, 
	\item $T_o$ - размер обучающей выборки,
	\item $T_e$ - размер тестовой (экзаменационной) выборки,
	\item $\hat{f}$ - точность построенной нейронной сети,
	\item $l$ - потери при обучении,
	\item $\Psi$ - время обучения нейронной сети в секундах,
	\item $\Sigma$ - количество эпох обучения.
\end{itemize}

\begin{table}[H]
	\begin{tabular}{|l|l|l|l|l|l|l|l|l|}
		\hline
Модель   & $N$&$NP$& $T_o$       & $T_e$  & $\hat{f}$ & l   & $\Sigma$ & $\Psi$ \\ \hline
$ f1-0 $  & [64, 64] & 6144 & 10240 & 1024 & 16.0000 & 0.0024 & 15000 & 45.9182 \\ \hline
$ f1-0 $  & [64, 64, 64] & 10240 & 10240 & 1024 & 16.0000 & 0.0051 & 15000 & 70.5227 \\ \hline
$ f1-7 $  & [64, 64] & 6144 & 10240 & 1024 & 16.0000 & 0.0053 & 15000 & 47.3304 \\ \hline
$ f1-7 $  & [64, 64, 64] & 10240 & 10240 & 1024 & 16.0000 & 0.0191 & 15000 & 67.8144 \\ \hline
$ f2-0 $  & [64, 64] & 6144 & 10240 & 1024 & 14.0000 & 0.1135 & 15000 & 287.8751 \\ \hline
$ f2-0 $  & [64, 64, 64] & 10240 & 10240 & 1024 & 13.0000 & 0.1594 & 15000 & 415.4322 \\ \hline
$ f2-7 $  & [64, 64] & 6144 & 10240 & 1024 & 14.0000 & 0.1007 & 15000 & 295.2384 \\ \hline
$ f2-7 $  & [64, 64, 64] & 10240 &10240 & 1024 & 14.0000 & 0.1126 & 15000 & 410.2425 \\ \hline
$ f3-0 $  & [64, 64] & 6144 & 10240 & 1024 & 12.0000 & 0.1446 & 15000 & 294.4967 \\ \hline
$ f3-0 $  & [64, 64, 64] & 10240 & 10240 & 1024 & 13.0000 & 0.1290 & 15000 & 416.2675 \\ \hline
$ f3-7 $  & [64, 64] & 6144 &10240 & 1024 & 10.0000 & 0.2218 & 15000 & 296.4944 \\ \hline
$ f3-7 $  & [64, 64, 64] & 10240 &10240 & 1024 & 10.0000 & 0.2152 & 15000 & 424.1930 \\ \hline
$ f4-0 $  & [64, 64] & 6144 & 10240& 1024 & 12.0000 & 0.1292 & 15000 & 306.2740 \\ \hline
$ f4-0 $  & [64, 64, 64] & 10240 & 10240 & 1024 & 12.0000 & 0.1259 & 15000 & 434.2595 \\ \hline
$ f4-7 $  & [64, 64] & 6144 & 10240 & 1024 & 10.0000 & 0.2403 & 15000 & 312.7606 \\ \hline
$ f4-7 $  & [64, 64, 64] & 10240 & 10240 & 1024 & 10.0000 & 0.2371 & 15000 & 447.1662 \\ \hline
$ f5-0 $  & [64, 64] & 6144 & 10240 & 1024 & 13.0000 & 0.1258 & 15000 & 319.0132 \\ \hline
$ f5-0 $  & [64, 64, 64] & 10240 & 10240 & 1024 & 12.0000 & 0.1264 & 15000 & 457.8168 \\ \hline
$ f5-7 $  & [64, 64] & 6144 & 10240 & 1024 & 9.0000 & 0.2458 & 15000 & 321.5454 \\ \hline
$ f5-7 $  & [64, 64, 64] & 10240 & 10240 & 1024 & 9.0000 & 0.2462 & 15000 & 465.3593 \\ \hline
$ f6-0 $  & [64, 64] & 6144 & 10240 & 1024 & 12.0000 & 0.1261 & 15000 & 321.7591 \\ \hline
$ f6-0 $  & [64, 64, 64] & 10240 & 10240 & 1024 & 12.0000 & 0.1261 & 15000 & 460.5911 \\ \hline
$ f6-7 $  & [64, 64] & 6144 & 10240 & 1024 & 9.0000 & 0.2491 & 15000 & 331.5827 \\ \hline
$ f6-7 $  & [64, 64, 64] & 10240 & 10240 & 1024 & 9.0000 & 0.2499 & 15000 & 475.0901 \\ \hline
$ f7-0 $  & [64, 64] & 6144 & 10240 & 1024 & 12.0000 & 0.1259 & 15000 & 273.3749 \\ \hline
$ f7-0 $  & [64, 64, 64] & 10240 & 10240 & 1024 & 12.0000 & 0.1265 & 15000 & 421.1694 \\ \hline
$ f7-7 $  & [64, 64] & 6144 & 10240 & 1024 & 9.0000 & 0.2524 & 15000 & 258.6057 \\ \hline
$ f7-7 $  & [32, 64, 32] & 5120 & 10240 & 1024 & 9.0000 & 0.2515 & 15000 & 247.9778 \\ \hline
$ f7-7 $  & [64, 64, 64] & 10240 & 10240 & 1024 & 9.0000 & 0.2527 & 15000 & 430.1847 \\ \hline
$ f8-0 $  & [64, 64] & 6144 & 10240 & 1024 & 12.0000 & 0.1262 & 15000 & 226.1201 \\ \hline
$ f8-0 $  & [64, 64, 64] & 10240 & 10240 & 1024 & 12.0000 & 0.1265 & 15000 & 332.4553 \\ \hline
$ f8-7 $  & [64, 64] & 6144 & 10240 & 1024 & 9.0000 & 0.2523 & 15000 & 225.5158 \\ \hline
$ f8-7 $  & [64, 64, 64] & 10240 & 10240 & 1024 & 9.0000 & 0.2544 & 15000 & 329.9783 \\ \hline
		\hline
	\end{tabular}
	\caption{Сравнение построенных нейронных сетей}
	\label{diff_f}
\end{table}
\bigskip

Из представленных выше результатов (графики и таблица \ref{diff_f}), можно сделать вывод,
нейронные сети прямого распространения с многослойным перцептроном, позволяют аппроксимировать криптографические преобразования.
Из этого можно сделать вывод, что такой метод криптоанализа можно использовать для оценки стойкости криптосистем.
%  $ f1-0 $ & NN & 0 & 256 & - & - & 12.3248 & 0.1242 & 10000 & 7.6048 \\ \hline
%  $ f1-0 $ & MNN & [64] & 2048 & - & - & 16.0000 & 0.0236 & 5000 & 7.4203 \\ \hline
%  $ f1-7 $ & NN & 0 & 256 & - & - & 12.7883 & 0.1165 & 20000 & 17.0164 \\ \hline
%  $ f1-7 $ & MNN & [64] & 2048 & - & - & 16.0000 & 0.0346 & 5000 & 9.3334 \\ \hline
%  $ f2-0 $ & NN & 0 & 256 & - & - & 7.9660 & 0.2527 & 5000 & 19.1589 \\ \hline
%  $ f2-0 $ & NN & 0 & 256 & - & - & 7.9660 & 0.2527 & 5000 & 21.9126 \\ \hline
%  $ f2-0 $ & MNN & [64] & 2048 & - & - & 13.0000 & 0.1712 & 10000 & 122.4510 \\ \hline
%  $ f2-7 $ & NN & 0 & 256 & - & - & 9.0365 & 0.2403 & 25000 & 23.4440 \\ \hline
%  $ f2-7 $ & MNN & [64] & 2048 & - & - & 13.0000 & 0.1518 & 10000 & 19.3323 \\ \hline
  