\documentclass[a4paper,12pt,twoside]{article}
\usepackage[T1,T2A]{fontenc}
\usepackage[utf8]{inputenc}


\usepackage{graphicx}
\usepackage{blindtext}
\usepackage{ mathrsfs }
\usepackage[russian,english]{babel}
\usepackage{enumitem}
\usepackage[english,russian]{babel}
\usepackage{amsmath}
\usepackage[top=20mm, bottom=20mm, left=20mm, right=20mm]{geometry}

\begin{document}
	\section{Однослойная сеть}
	
	Для оценки точности построенной модели использовалось расстояние Хэмминга.
	
	$f(x, y) = \sum x_i \oplus y_i$

	\bigskip

	\noindent  Для обозначения задачи использовались следующие функции:
	
	$gost(x)$ - итерация ГОСТ 28147,
		
		\bigskip
		
	$LE(x) = x_1$, где $x = x_1 || x_2$.
	
	\bigskip
	\noindent Рассматривались следующие частные случаи поставленной задачи:
	
	\begin{enumerate}
		\item $x = L || R,$ где $L = 0^{32}$, $y=LB(gost(x)),$ где $y \in V^{32}$
		\item $x = L || R,$ где $R = 0^{32}$, $y=LB(gost(x)),$ где $y \in V^{32}$
		\item $x = L || R,$ где $R = 0^{32}$, $y=gost(x),$ где  $y \in V^{64}$
		\item $x = L || R$, $y=LB(gost(x)),$ где  $y \in V^{32}$
		\item $x = L || R$, $y=gost(x),$ где $y \in V^{64}$
		\item $x = L || R$, $y=gost^2(x),$ где $y \in V^{64}$
		\item $x = L || R$, $y=gost^3(x),$ где $y \in V^{64}$
	\end{enumerate}
	
	\noindent\textbf{Результаты:}
	
	\bigskip	
	
		\begin{table}[ht!]
			\begin{tabular}{ll}
				Случай &  Расстояние \\
				1. &  $f(pred, real) = 0$\\
				2. &  $f(pred, real) = 7$\\
				3. &  $f(pred, real) = 7$\\
				4. &  $f(pred, real) = 11$\\
				5. &  $f(pred, real) = 11$\\
				6. &  $f(pred, real) = 27$\\
				7. &  $f(pred, real) = 31$\\
			\end{tabular}
		\end{table}
	
	\section{Многослойная сеть}
	\bigskip
	Рассматривались следующие частные случаи поставленной задачи:
	
	\begin{enumerate}
		\item $x = L || R$, $y=gost(x),$ где $y \in V^{64}$
		\item $x = L || R$, $y=gost^2(x),$ где $y \in V^{64}$
		\item $x = L || R$, $y=gost^3(x),$ где $y \in V^{64}$
	\end{enumerate}

		\begin{table}[ht!]
		\begin{tabular}{lll}
			Случай &  Расстояние & Кол-во слоев\\
			1. &  $f(pred, real) = 7$ & 1\\
			1. &  $f(pred, real) = 4$ & 2\\
			2. &  $f(pred, real) = 29$ & 1\\
			2. &  $f(pred, real) = 30$ & 2\\
			3. &  $f(pred, real) = 31$ & 1\\
			3. &  $f(pred, real) = 31$ & 2\\
			3. &  $f(pred, real) = 31$ & 3\\
		\end{tabular}
	\end{table}
	
\end{document}