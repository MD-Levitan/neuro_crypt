\documentclass[a4paper,12pt,twoside]{article}
\usepackage[T1,T2A]{fontenc}
\usepackage[utf8]{inputenc}


\usepackage{graphicx}
\usepackage{blindtext}
\usepackage{ mathrsfs }
\usepackage[russian,english]{babel}
\usepackage{enumitem}
\usepackage[english,russian]{babel}
\usepackage{amsmath}
\usepackage{amssymb}
\usepackage[top=20mm, bottom=20mm, left=20mm, right=20mm]{geometry}

\begin{document}
	\section{Применение нейроных сетей для апроксимации\\ криптографического примитива ГОСТ 28147}
	
	Для оценки точности построенной модели использовалось расстояние Хэмминга.
	
	$w(y, \hat{y}) = \sum_{i=1}^{j} y_i \oplus \hat{y}_i$, где $y_i \in V_j$.

	\bigskip

	\noindent  В данной работе рассмотривались следующие задачи:
	\begin{enumerate}
	\item 
	$y = g_0(x, x_1) \equiv x \oplus x_1,$ где
	
	$x \in V_4$ - вектор входных данных,
	
	$x_1 \in V_4$ - добавочный вектор данных;
	\bigskip
		
	\item 
	$y = g_1(x, x_1, k) \equiv S[x \boxplus k] \oplus x_1,$ где
	
	 $x \in V_4$ - вектор входных данных,
	 
	 $x_1 \in V_4$ - добавочный вектор данных,
	 
	 $k \in V_4$ - ключ (4-битная часть ключа),
	 
	 $S$ - стандартный S-блок из ГОСТ-28147;
	\bigskip
	
	\item 
	$y = g_2(\hat{x}, x_1, K) \equiv S[\hat{x}] \oplus x_1,$ где
	
	$x \in V_4$ - вектор входных данных,
	
	$x_1 \in V_4$ - добавочный вектор данных,
	
	$K=k_1 || k_2 || ... ||k _8, k_i \in V_{4}$ - ключ,
	
	$X=x^{(1)} || x^{(2)} || ... || x^{(8)}, x^{(i)} \in V_{4}$ - блок входных данных,
	
	$S$ - стандартный S-блок из ГОСТ-28147,
	
	$\hat{X} = \bar{X} \boxplus \bar{K} \equiv \hat{x}^{(1)} || \hat{x}^{(2)} || ... || \hat{x}^{(8)}, \hat{x}^{(i)} \in V_{4}$.
	
	\bigskip
	\end{enumerate}
		
	\bigskip
	\noindent Для решения поставленных задач использовались следующие математические модели:
	
	\begin{enumerate}
		\item Однослойная нерйоная сеть, 15000 итераций обучения.
		%\item Однослойная нерйоная сеть, 15000 итераций обучения.
		
		\item Многослойная нерйоная сеть с одним скрытым слоем с 4 нейронами на скрытом слое, 15000 итераций обучения.
		\item Многослойная нерйоная сеть с одним скрытым слоем с 8 нейронами на скрытом слое, 15000 итераций обучения.
		\item Многослойная нерйоная сеть с одним скрытым слоем с 16 нейронами на скрытом слое, 15000 итераций обучения.
		\item Многослойная нерйоная сеть с одним скрытым слоем с 32 нейронами на скрытом слое, 15000 итераций обучения.
		\item Многослойная нерйоная сеть с двумя скрытым слоями с 32 нейронами на скрытых слоях, 15000 итераций обучения.
	\end{enumerate}
	
	\bigskip
	\noindentКомпьютерные эксперименты проводились на следующих данных:
	\begin{enumerate}
		\item Обучающая выборка $T_o=18 * 10^3$ пар $(x, x_1)$;
		\item Экзаменационная выборка $T_e=2 * 10^3$ пар $(x, x_1)$.
	\end{enumerate}
	
	\bigskip
	\noindent
	Для оценки точности использовалась следующая функция:
	$\hat{f} = \dfrac{1}{T_e}\sum_{j=1}^{T_e}w(y^{(j)}, \hat{y^{(j)}})$.
	
	\bigskip
	
	\noindent\textbf{Результаты:}
	
	\bigskip	
	
		\begin{table}[ht!]
			\begin{tabular}{lll}
				Задача &  Модель & Результаты \\
				1. & 1 & $f(pred, real) = 1.3699500560760498$\\
				1. & 2 & $f(pred, real) = 1.4635499715805054$\\
				1. & 3 & $f(pred, real) = 0.8504499793052673$\\
				1. & 4 & $f(pred, real) = 0.07020000368356705$\\
				1. & 5 & $f(pred, real) = 0.001$\\
				1. & 6 & $f(pred, real) = 2.012$\\
				
				2. & 1 & $f(pred, real) = 1.6059999465942383$\\
				2. & 2 & $f(pred, real) = 1.4520000219345093$\\
				2. & 3 & $f(pred, real) = 1.253999948501587$\\
				2. & 4 & $f(pred, real) = 1.1009999513626099$\\
				2. & 5 & $f(pred, real) = 0.9789999723434448$\\
				2. & 6 & $f(pred, real) = 2.001212$\\
				
				3. & 1 & $f(pred, real) = 1.6059999465942383$\\
				3. & 2 & $f(pred, real) = 1.4520000219345093$\\
				3. & 3 & $f(pred, real) = 1.253999948501587$\\
				3. & 4 & $f(pred, real) = 1.1009999513626099$\\
				3. & 5 & $f(pred, real) = 0.9789999723434448$\\
				3. & 6 & $f(pred, real) = 2.001212$\\
			
			\end{tabular}
		\end{table}
	
\end{document}